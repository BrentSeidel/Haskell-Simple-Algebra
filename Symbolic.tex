\documentclass[10pt, openany]{book}

\usepackage{fancyhdr}
\usepackage{multind}
\usepackage{amsmath}
\usepackage{amsfonts}
\usepackage{geometry}
\geometry{letterpaper}

\usepackage{fancyvrb}
\usepackage{fancybox}
%
% Rules to allow import of graphics files in EPS format
%
\usepackage{graphicx}
\DeclareGraphicsExtensions{.eps}
\DeclareGraphicsRule{.eps}{eps}{.eps}{}
%
% Define rules to format and embed Haskell code in the text
%
\usepackage{listings}
\lstloadlanguages{Haskell}
\lstnewenvironment{code}
    {\lstset{}%
      \csname lst@SetFirstLabel\endcsname}
    {\csname lst@SaveFirstLabel\endcsname}
    \lstset{
      basicstyle=\small\ttfamily,
      flexiblecolumns=false,
      basewidth={0.5em,0.45em},
      literate={+}{{$+$}}1 {/}{{$/$}}1 {*}{{$*$}}1 {=}{{$=$}}1
               {>}{{$>$}}1 {<}{{$<$}}1 {\\}{{$\lambda$}}1
               {\\\\}{{\char`\\\char`\\}}1
               {->}{{$\rightarrow$}}2 {>=}{{$\geq$}}2 {<-}{{$\leftarrow$}}2
               {<=}{{$\leq$}}2 {=>}{{$\Rightarrow$}}2 
               {\ .}{{$\circ$}}2 {\ .\ }{{$\circ$}}2
               {>>}{{>>}}2 {>>=}{{>>=}}2
               {|}{{$\mid$}}1
               {/=}{{$ \ne$}}2
    }
%
% For requirements trace tags
%
\makeindex{tags}
\newcommand{\paratagname}{empty}
\newcommand{\paratag}[1]{\renewcommand{\paratagname}{#1}{\begin{flushright}{\{\{#1\}\}}\end{flushright}}\index{tags}{#1}}
\newcommand{\shalltag}[1]{\textbf{shall} {\tiny \{\{\paratagname\ #1\}\}}\index{tags}{\paratagname!\paratagname\ #1}}
%
%  Other commands
%
\newcommand{\settitle}[1]{\newcommand{\gettitle}{#1}\title{#1}}
%
% Setup page headers and footers
%
\fancypagestyle{plain}{
\fancyhf{}
\fancyhead[L]{\gettitle}
\chead{}
\rhead{}
\fancyfoot[R]{Page \thepage}
\renewcommand{\headrulewidth}{0.4pt}
\renewcommand{\footrulewidth}{0.4pt}}

\pagestyle{fancy}
\fancyhead{}
\lhead{\gettitle}
\chead{}
\rhead{}
\cfoot{}
\rfoot{Page \thepage}
\renewcommand{\headrulewidth}{0.4pt}
\renewcommand{\footrulewidth}{0.4pt}
%
% Setup paragraph formatting - no indentation with a blank line between paragraphs.
%
\setlength{\parindent}{0pt}
\setlength{\parskip}{1ex plus 0.5ex minus 0.2ex}
%
% Front Matter
%
\settitle{Software Detail Design Document for Symbolic Math Package}
\author{Brent Seidel \\ Phoenix, AZ}
\date{ \today }
%========================================================
%%% BEGIN DOCUMENT
\begin{document}
%
% Produce the front matter
%
\frontmatter
\maketitle
This document is \copyright 2009 by Brent Seidel.  The software included in this document was written as a learning exercise and is probably not very good Haskell code.  It may be freely distributed and used for any purpose whatsoever.  If you need someone who knows a little Haskell, contact me.
\tableofcontents
\mainmatter
%--------------------------------------------------------------------------------------------------
\chapter{Polynomial Utilities}
\section{Program Headers}
\paratag{SWDDD POLY HEADER}
This file and all items defined in it are contained in the module ``Polynomial''.  Eventually, a data type and class for Polynomial should be added.

\subsection{Requirements}
\begin{itemize}
\item The program \shalltag{10} be in a module named ``Polynomial''.
\item The functions polyEval, polyAdd, polyMul, polyIntegrate, polyDerive, and polyShow \shalltag{20} be exported from the module.
\end{itemize}

\subsection{Header Code}
\begin{code}
module Polynomial ( polyEval, polyAdd, polyMul, polyIntegrate, polyDerive, polyShow ) where
\end{code}

%--------------------------------------------------------------------------------------------------
\section{Evaluate a Polynomial Using Newton's Method}
\paratag{SWRDD POLY EVAL}
This function evaluates a polynomial at a specific value using Newton's method.  A polynomial of standard form:
\[
p(X) = a_nX^n + a_{n-1}X^{n-1} \cdots + a_2X^2 + a_1X + a_0
\]

is rewriten as:
\[
p(X) = ((((a_nX) + a_{n-1})X) + \cdots + a_2)X + a_1)X + a_0
\]
Evaluating the polynomial in this fashion also minimizes the number of multiplications required.

\subsection{Requirements}
\begin{itemize}
\item The newton function \shalltag{10} evaluate a polynomial at a specified value.
\item The polynomial \shalltag{20} be stored in a list containing the $a_n$ coefficients.
\item The coefficients in the list \shalltag{30} be arranged with $a_0$ first.
\end{itemize}

\subsection{Function Definition}
This function evaluates a polynomial using Newton's method.
\begin{code}
polyEval :: Num a => [a] -> a -> a
polyEval [] a = 0
polyEval (x:xs) y = x + y * polyEval xs y
\end{code}

%--------------------------------------------------------------------------------------------------
\section{Add Two Polynomials}
\paratag{SWRDD POLY ADD}
This function adds two polynomials of order $a_1$ and $a_2$.
\[
P_1(X) = \sum_{n = 0}^{a_1} c_nX^n
\]
and
\[
P_2(X) = \sum_{n = 0}^{a_2} d_nX^n
\]
the sum is given by
\[
P_1(X) + P_2(X)= \sum_{n = 0}^{MAX (a_1, a_2)} (c_n + d_n)X^n
\]

\subsection{Requirements}
\begin{itemize}
\item The sum of two polynomials \shalltag{10} be the sum of the corresponding coefficients.
\item Unmatched coefficients \shalltag{20} be passed unchanged to the output.
\end{itemize}

\subsection{Function Definition}
This function adds two polynomials producing a third.
\begin{code}
polyAdd :: (Num a) => [a] -> [a] -> [a]
polyAdd [] [] = []
polyAdd x []  = x
polyAdd [] x  = x
polyAdd (x:xs) (y:ys) = x + y :  polyAdd xs ys
\end{code}

%--------------------------------------------------------------------------------------------------
\section{Multiply Two Polynomials}
\paratag{SWRDD POLY MULTIPLY}
This function multiplies two polynomials of order $a_1$ and $a_2$.  Thus:
\[
P_1(X) = \sum_{i = 0}^{a_1}c_iX^i, P_2(X) = \sum_{j = 0}^{a_2}d_jX^j
\]
The product then is:
\[
P_1(X) \times P_2(X) = \sum_{i = 0}^{a_1}c_iX^i \times \sum_{j = 0}^{a_2}d_jX^j = \sum_{i = 0}^{a_1} \sum_{j = 0}^{a_2} c_i d_j X^{i + j}
\]

\subsection{Requirements}
\begin{itemize}
\item The product of two polynomials \shalltag{10} be the sum of the the products of all the monomials.
\end{itemize}

\subsection{Function Definition}
This function multiplies two polynomials producing a third.  This includes a function that multiplies all the coefficients in a polynomial by a scalar value.
\begin{code}
polyScale :: (Num a) => [a] -> a -> [a]
polyScale [] _ = []
polyScale _ 0  = []
polyScale xs y = [x*y | x<-xs]

polyMul :: (Num a) => [a] -> [a] -> [a]
polyMul [] [] = []
polyMul _ []  = []
polyMul [] _  = []
polyMul (x:xs) y = polyAdd (polyScale y x) (polyMul xs (0 : y))
\end{code}

%--------------------------------------------------------------------------------------------------
\section{Integrate a Polynomial}
\paratag{SWRDD POLY INTEGRATE}
The integral of a monomial with respect to $X$, $c_{n}X^n$ is given by $\frac{c_n}{n+1}X^{n+1}$.  A polynomial is simply a sum of monomials.  If
\[
P(X) = \sum_{n = 0}^{a} c_nX^n
\]
then
\[
\int P(X) dX = \sum_{n = 0}^{a}\int c_nX^n dX\\
= \sum_{n = 0}^{a} \frac{c_n}{n + 1}X^{n+1}
\]

\subsection{Requirements}
\begin{itemize}
\item The integral of a polynomial \shalltag{10} shall be the sum of the integral of each monomial making up the polynomial.
\item The integral of the monomial $c_{n}X^n$ \shalltag{20} be $\frac{c_{n}}{n+1}X^{n+1}$.
\item The constant term in the result \shalltag{30} be set to zero.
\end{itemize}

\subsection{Function Definition}
Note that since division is involved, the coefficients need to be of the fractional type.
\begin{code}
polyIntegrate :: (Fractional a) => [a] -> [a]
polyIntegrate [] = []
polyIntegrate (x:xs) = 0 : x : helper xs 2
  where
    helper :: (Fractional a) => [a] -> a -> [a]
    helper (x:xs) n
      | xs == []  = [x / n]
      | otherwise = (x / n) : helper xs (n + 1)
\end{code}

%--------------------------------------------------------------------------------------------------
\section{Find the Derivative of a Polynomial}
\paratag{SWRDD POLY DERIVATIVE}
The derivative of a monomial with respect to $X$, $c_{c}X^n$ is given by $nc_{n}X^{n-1}$.  A polynomial is simply a sum of monomials.

\subsection{Requirements}
\begin{itemize}
\item The derivative of a polynomial \shalltag{10} shall be the sum of the derivative of each monomial making up the polynomial.
\item The derivative of a constant monomial \shalltag{20} be zero.
\item The derivative of the monomial $c_{n}X^n$ \shalltag{30} be $nc_{n}X^{n-1}$.
\end{itemize}

\subsection{Function Definition}
\begin{code}
polyDerive :: (Num a) => [a] -> [a]
polyDerive [] = []
polyDerive (x:xs)
  | xs == []  = [0]
  | otherwise = helper xs 1
  where
    helper :: (Num a) => [a] -> a -> [a]
    helper (x:xs) n
      | xs == []  = [n * x]
      | otherwise = (n * x) : helper xs (n + 1)
\end{code}

%--------------------------------------------------------------------------------------------------
\section{Display a Polynomial}
\paratag{SWRDD POLY DISPLAY}
This function converts a polynomial in standard form to a string.  Polynomials are stored as a list of coefficients.

\subsection{Requirements}
\begin{itemize}
\item The coefficients of a polynomials \shalltag{10} be extracted from a list.
\item The coefficients in the list \shalltag{20} be arranged with the lowest order coefficient first.
\item If the coefficient is zero, that term \shalltag{30} be omitted.
\item If the coefficient is one, the coefficient \shalltag{40} be omitted.
\end{itemize}

\subsection{Function Definition}
\begin{code}
polyShow :: (Num a, Ord a) => [a] -> String
polyShow [] = ""
polyShow (x:xs)
  | x == 0        = helper xs 1
  | x > 0         = helper xs 1 ++ " + " ++ show x
  | x < 0         = helper xs 1 ++ " - " ++ show (-x)
  where 
    helper :: (Num a, Ord a) => [a] -> Int -> String
    helper [] a = ""
    helper (x:xs) n
      | x == 0              = helper xs (n + 1)
      | x == 1 && xs == []  = helper xs (n + 1) ++ "X^" ++ show n
      | x > 0 && xs ==[]    = helper xs (n + 1) ++  show x ++ "X^" ++ show n
      | x < 0 && xs == []   = helper xs (n + 1) ++ show x ++ "X^" ++ show n
      | x == 1              = helper xs (n + 1) ++ " + " ++ "X^" ++ show n
      | x > 0               = helper xs (n + 1) ++ " + " ++ show x ++ "X^" ++ show n
      | x < 0               = helper xs (n + 1) ++ " - " ++ show (-x) ++ "X^" ++ show n
\end{code}

%--------------------------------------------------------------------------------------------------
\chapter{Parsing Utilities}
\section{Program Headers}
\paratag{SWDDD PARSE HEADER}
This file and all items defined in it are contained in the module ``Parse''.

\subsection{Requirements}
\begin{itemize}
\item The program \shalltag{10} be in a module named ``Parse''.
\item The Char module \shalltag{20} be imported to provide character functions.
\item Only the parsePoly function \shalltag{30} be exported.
\end{itemize}

\subsection{Header Code}
\begin{code}
module Parse (parsePoly) where

import Char (isDigit, isSpace)
\end{code}

%--------------------------------------------------------------------------------------------------
\section{Get Coefficient}
\paratag{SWDDD PARSE POLY COEF}
This function is part of the tokenizer and is used to extract a coefficient from the input string.  The coefficient should be a valid double precision number.
\subsection{Requirments}
\begin{itemize}
\item A coefficient \shalltag{10} be a double precision number.
\item If a coefficient is omitted, its value \shalltag{20} be set to 1.
\item If the sign is omitted, it \shalltag{30} be assumed to be positive.
\item Whitespace characters \shalltag{40} be ignored.
\item The digits before or after the decimal point \shalltag{50} be optional.
\item The decimal point \shalltag{60} be optional.
\item The exponent \shalltag{70} be optional.
\item If the exponent is present, it \shalltag{80} consist of the letter 'E' followed by an optional sign and non-optional digits.
\item The coefficient \shalltag{90} be terminated by the end of the input string, a sign, the letter 'X', or an error condition.
\end{itemize}

\subsection{Function Definitions}

The coefficient parser is implemented using a state machine.  Some helper functions are also defined.
\subsubsection{Define Support}
Define a data type for the states and some helper functions.
\begin{code}
data GetCoefStates = StCoefStart | StCoefFoundSign | StCoefPreDecimal | StCoefPostDecimal | StCoefFoundExp | StCoefExpSign | StCoefExpValue

isSign :: Char -> Bool
isSign x = (x == '+') || (x == '-')

isExp :: Char -> Bool
isExp x = (x == 'e') || (x == 'E')

isX :: Char -> Bool
isX x = (x == 'x') || (x == 'X')
\end{code}
\subsubsection{Define the Main Coefficient Parser}
The parser takes a state and a string as inputs and returns a tuple consisting of two strings.  The first string is the parsed data and the second string is the remaining data.

If no input string is passed, two null strings are returned.
\begin{code}
getCoefficient :: GetCoefStates -> String -> (String, String)
getCoefficient _ [] = ([], [])
\end{code}
\subsubsection{The Start State}
In the first state, we can take a digit, a sign, a decimal point, or an ``X''.
\begin{code}
getCoefficient StCoefStart (x:xs)
  | isSpace x = (skip1, skip2)
  | isDigit x = (x : dig1, dig2)
  | x == '-'  = (x : sign1, sign2)
  | x == '+'  = (sign1, sign2)
  | isX x     = ("1", x : xs)
  | x == '.'  = (x : dig3, dig4)
  | otherwise = ("*Error", "*Error")
    where
      (skip1, skip2) = getCoefficient StCoefStart xs
      (dig1, dig2) = getCoefficient StCoefPreDecimal xs
      (dig3, dig4) = getCoefficient StCoefPostDecimal xs
      (sign1, sign2) = getCoefficient StCoefFoundSign xs
\end{code}
\subsubsection{The Found Sign State}
Once a sign has been found, signs are no longer valid characters.
\begin{code}
getCoefficient StCoefFoundSign (x:xs)
  | isSpace x = (skip1, skip2)
  | isDigit x = (x : dig1, dig2)
  | x == '.'  = (x : dig3, dig4)
  | isX x     = ("1", x : xs)
  | otherwise = ("*Error", "*Error")
    where
      (skip1, skip2) = getCoefficient StCoefFoundSign xs
      (dig1, dig2) = getCoefficient StCoefPreDecimal xs
      (dig3, dig4) = getCoefficient StCoefPostDecimal xs
\end{code}
\subsubsection{The Pre-Decimal Point State}
Now, process the digits before the decimal point.
\begin{code}
getCoefficient StCoefPreDecimal (x:xs)
  | isSpace x = (dig1, dig2)
  | isDigit x = (x : dig1, dig2)
  | isExp x   = (x : exp1, exp2)
  | isX x     = ([], x : xs)
  | isSign x  = ([], x : xs)
  | x == '.'  = (x : dec1, dec2)
  | otherwise = ("*Error", "*Error")
    where
      (dig1, dig2) = getCoefficient StCoefPreDecimal xs
      (dec1, dec2) = getCoefficient StCoefPostDecimal xs
      (exp1, exp2) = getCoefficient StCoefFoundExp xs
\end{code}
\subsubsection{The Post-Decimal Point State}
If a decimal point has been found, process any digits after it.
\begin{code}
getCoefficient StCoefPostDecimal (x:xs)
  | isSpace x = (dig1, dig2)
  | isDigit x = (x : dig1, dig2)
  | isExp x   = (x : exp1, exp2)
  | isX x     = ([], x : xs)
  | isSign x  = ([], x : xs)
  | otherwise = ("*Error", "*Error")
    where
      (dig1, dig2) = getCoefficient StCoefPostDecimal xs
      (exp1, exp2) = getCoefficient StCoefFoundExp xs
\end{code}
\subsubsection{The Found Exponent State}
If an ``E'' has been found, look for the exponent.
\begin{code}
getCoefficient StCoefFoundExp (x:xs)
  | isSpace x = (exp1, exp2)
  | isDigit x = (x : dig1, dig2)
  | isSign x  = (x : sign1, sign2)
  | otherwise = ("*Error", "*Error")
    where
      (exp1, exp2) = getCoefficient StCoefFoundExp xs
      (sign1, sign2) = getCoefficient StCoefExpSign xs
      (dig1, dig2) = getCoefficient StCoefExpValue xs
\end{code}
\subsubsection{The Exponent Sign State}
The exponent should have a sign.
\begin{code}
getCoefficient StCoefExpSign (x:xs)
  | isSpace x = (sign1, sign2)
  | isDigit x = (x : dig1, dig2)
  | otherwise = ("*Error", "*Error")
    where
      (sign1, sign2) = getCoefficient StCoefExpSign xs
      (dig1, dig2) = getCoefficient StCoefExpValue xs
\end{code}
\subsubsection{The Exponent Value State}
The exponent value should consist only of digits.
\begin{code}
getCoefficient StCoefExpValue (x:xs)
  | isSpace x = (dig1, dig2)
  | isDigit x = (x : dig1, dig2)
  | isSign x  = ([], x : xs)
  | isX x     = ([], x : xs)
  | otherwise = ("*Error", "*Error")
    where
      (dig1, dig2) = getCoefficient StCoefExpValue xs
\end{code}

%--------------------------------------------------------------------------------------------------
\section{Get Exponent}
\paratag{SWDDD PARSE POLY EXP}
This function is part of the tokenizer and is used to extract an exponent from the input string.  The exponent should be a positive integer.
\subsection{Requirments}
\begin{itemize}
\item An exponent \shalltag{10} be a positive integer.
\item If an exponent is omitted and the 'X' is present, its value \shalltag{20} be set to 1.
\item If both the exponent and the 'X' are omitted, its value \shalltag{30} be set to 0.
\item Whitespace characters \shalltag{40} be ignored.
\item The exponent \shalltag{90} be terminated by the end of the input string, a sign, or an error condition.
\end{itemize}

\subsection{Function Definitions}
The exponent parser is similar to the coefficient parser, but simpler.  All it needs to do is to look for the independent variable, ``X'', the exponent symbol, and then a string of digits.
\subsubsection{Define Support}
A data type for the states is defined.
\begin{code}
data GetExpStates = StExpStart | StExpFoundX | StExpFoundPow | StExpDigits
\end{code}
\subsubsection{Define the Exponent Parser}
The exponent parser definition is similar to the coefficient parser.  It takes a state and a string and returns two string.  If, in the start state, the input string is null, an exponent of zero is returned.  If the input string is null, two null strings are returned.
\begin{code}
getExponent :: GetExpStates -> String -> (String, String)

getExponent StExpStart [] = ("0", [])
getExponent StExpFoundX [] = ("1", [])
getExponent _ [] = ([], [])
\end{code}
\subsubsection{The Start State}
In this state, an ``X'' is expected.  If a sign is found, then there is no independent variable and the exponent is zero.
\begin{code}
getExponent StExpStart (x:xs)
  | isSpace x = (skip1, skip2)
  | isSign x  = ("0", x : xs)
  | isX x     = (dig1, dig2)
  | otherwise = ("*Error", "*Error")
    where
      (skip1, skip2) = getExponent StExpStart xs
      (dig1, dig2) = getExponent StExpFoundX xs
\end{code}
\subsubsection{The Found X State}
In this state, we should either find a sign or an exponent symbol.
\begin{code}
getExponent StExpFoundX (x:xs)
  | isSpace x    = (skip1, skip2)
  | x == '^'     = (pow1, pow2)
  | isSign x     = ("1", x : xs)
  | otherwise    = ("*Error", "*Error")
    where
      (skip1, skip2) = getExponent StExpFoundX xs
      (pow1, pow2) = getExponent StExpFoundPow xs
\end{code}
\subsubsection{The Found Power State}
Once the exponent symbol has been found, the only valid thing to follow is a digit.
\begin{code}
getExponent StExpFoundPow (x:xs)
  | isSpace x = (skip1, skip2)
  | isDigit x = (x : dig1, dig2)
  | otherwise = ("*Error", "*Error")
    where
      (dig1, dig2) = getExponent StExpDigits xs
      (skip1, skip2) = getExponent StExpFoundPow xs
\end{code}
\subsubsection{The Exponent Digits State}
Now, just accumulate all the digits until something else is found.
\begin{code}
getExponent StExpDigits (x:xs)
  | isSpace x = (dig1, dig2)
  | isDigit x = (x : dig1, dig2)
  | isSign x  = ([], x : xs)
  | otherwise = ("*Error", "*Error")
    where
      (dig1, dig2) = getExponent StExpDigits xs
\end{code}

%--------------------------------------------------------------------------------------------------
\section{Tokenizer}
\paratag{SWDDD PARSE POLY TOKEN}
The tokenizer reads through the input string and breaks it up into tokens.
\subsection{Requirments}
\begin{itemize}
\item A parsed polynomial string \shalltag{10} return a list of coefficients and exponents.
\item The coefficients \shalltag{20} be double precision floating point numbers.
\item The exponents \shalltag{30} be integers.
\item The independent variable \shalltag{40} be `X'.
\item If the exponent on `X' is omitted, it \shalltag{50} be assumed to be 1.
\item If the coefficient is not followed by the independent variable, the exponent \shalltag{60} be assumed to be 0.
\end{itemize}

\subsection{Function Definitions}
\begin{code}
tokenize :: String -> [(Double, Int)]
tokenize [] = []
tokenize x
  | rem1 == "*Error" = []
  | rem2 == "*Error" = []
  | otherwise        = (coeff, exp) : tokenize rem2
  where
  (coeff1, rem1) = getCoefficient StCoefStart x
  (exp1, rem2) = getExponent StExpStart rem1
  coeff = (read coeff1) :: Double
  exp = (read exp1) :: Int
\end{code}

%--------------------------------------------------------------------------------------------------
\section{Build}
\paratag{SWDDD PARSE POLY BUILD}
The build process takes a tokenized polynomial (a list of coefficients and exponents) and creates a polynomial array.
\subsection{Requirments}
\begin{itemize}
\item The build process \shalltag{10} take the tokenized polynomial and construct a polynomial array.
\item The polynomial array \shalltag{20} be an array of double precision numbers.
\item The polynomial array \shalltag{30} contain the coefficients.
\item The coefficients \shalltag{40} be located in the indices corresponding to the exponents.
\end{itemize}

\subsection{Function Definitions}
\begin{code}
maxToken :: Int -> [(Double, Int)] -> Int
maxToken x [] = x

maxToken x (y:ys)
  | x > t = maxToken x ys
  | otherwise = maxToken t ys
  where
    (_, t) = y  

zeros :: [Double]
zeros = 0.0 : zeros

updatePoly :: [(Double, Int)] -> [Double] -> [Double]
updatePoly [] x = x

updatePoly (x:xs) y = result
  where
    (coeff, index) = x
    beg = take index y
    end = drop (index + 1) y
    mid = y!!index + coeff
    result = updatePoly xs (beg ++ [mid] ++ end)

buildPoly :: [(Double, Int)] -> [Double]
buildPoly [] = []

buildPoly x = result
  where
    order = maxToken 0 x
    arry = take (order + 1) zeros
    result = updatePoly x arry
\end{code}

%--------------------------------------------------------------------------------------------------
\section{Parse}
\paratag{SWDDD PARSE POLY PARSE}
The polynomial parser combines the process of the tokenizer and the builder.
\subsection{Requirments}
\begin{itemize}
\item The parser \shalltag{10} convert a string containing a polynomial into an array of double precision numbers.
\end{itemize}

\subsection{Function Definitions}
\begin{code}
parsePoly :: String -> [Double]
parsePoly [] = []

parsePoly x = buildPoly (tokenize x)
\end{code}

\chapter{Zero Finding Methods}
\section{Program Headers}
\paratag{SWDDD ZERO HEADER}
This file and all items defined in it are contained in the module ``Solver''.  This module contains routines to solve equations.  In most cases this involves finding ``zeros'' of a function.

\subsection{Requirements}
\begin{itemize}
\item The program \shalltag{10} be in a module named ``Solver''.
\item The solver module \shalltag{20} import the Complex module.
\end{itemize}

\subsection{Header Code}
\begin{code}
module Solver where

import Complex
\end{code}

%--------------------------------------------------------------------------------------------------
\section{Bisection Method}
\paratag{SWRDD ZERO BISECT}
The bisection method starts with two guesses for the zero.  The function must evaluate to a positive value at one guess and a negative value at the other.  At each iteration, the middle value is tested to determine on which side the zero lies.  The interval is then reduced by half.  After a sufficient number of iterations or a sufficiently small interval is reached, the value is returned.

Note that each iteration produces one additional bit of the solution, so convergence is linear.

\subsection{Requirements}
\begin{itemize}
\item The bisection algorithm \shalltag{10} accept as input a function, a lower limit, and an upper limit.
\item The bisection algorithm \shalltag{20} return a value for the zero and a string containing a message.
\item The input function \shalltag{30} accept a numeric input and return a numeric result.
\item The value of the input function \shalltag{40} be checked at the upper and lower limit to ensure that the signs are opposite.
\item If the signs are not opposite, an error message \shalltag{50} be returned.
\item The iteration limit \shalltag{60} be set to 64.
\item If a zero is reached exactly, iteration \shalltag{70} be terminated and that value returned.
\end{itemize}

\subsection{Function Definition}
This function finds a zero of a function using the bisection method.
\begin{code}
bisect :: (Floating a, Ord a) => (a -> a) -> a -> a -> (a, String)
bisect f lower upper
  | (f lower) == 0             = (lower, "Zero found exactly at lower limit.")
  | (f upper) == 0             = (upper, "Zero found exactly at upper limit.")
  | (f lower) * (f upper) >= 0 = (0, "ERROR: Function has same sign at upper and lower limits.")
  | otherwise                  = bisectHelper f lower upper 64
  where
    bisectHelper :: (Floating a, Ord a) => (a -> a) -> a -> a -> Int -> (a, String)
    bisectHelper f lower upper limit
      | limit <= 0              = (mid, "Zero found between " ++ show lower ++ " and " ++ show upper)
      | (f mid) == 0            = (mid, "Zero found exactly with " ++ show limit ++ " iterations left to go.")
      | (f lower) * (f mid) < 0 = bisectHelper f lower mid (limit - 1)
      | otherwise               = bisectHelper f mid upper (limit - 1)
      where
        mid = (lower + upper) / 2
\end{code}

%--------------------------------------------------------------------------------------------------
\section{Newton's Method}
\paratag{SWRDD ZERO NEWTON}
Newton's method (aka Newton-Raphson method) is one of the better known root finding algorithms.  Its main drawback is that it depends on having the derivative of the function.  The basic iteration step is:
\[
p_{n+1} = p_n - \frac{f(p_n)}{f'(p_n)}
\]
Note that if $f'(p_n)$ becomes zero, the iteration cannot be continued.

When it works, it can achieve quadratic convergence.  However, it is not always guaranteed to work.
\subsection{Requirements}
\begin{itemize}
\item The Newton's method algorithm \shalltag{10} accept a function, its derivative, a starting guess, a tolerance, and number of iterations.
\item The Newton's method algorithm \shalltag{20} return a result and a message indicating the meaning of the result.
\item If the derivative of the function is equal to zero, iteration \shalltag{30} be terminated and an error message returned.
\item If the required tolerance is not reached before the allowed number of itereations, an error message \shalltag{40} be returned.
\end{itemize}

\subsection{Function Definition}
\begin{code}
newton :: (Floating a, Ord a) => (a -> a) -> (a -> a) -> a -> a -> Int ->(a, String)
newton f fP guess tol limit
  | fP guess == 0           = (guess, "ERROR: Derivative of function reached zero.  Cannot continue.")
  | abs(next - guess) < tol = (guess, "Solution found with " ++ show limit ++ " iterations remaining.")
  | limit <= 0              = (guess, "ERROR: No solution within tolerance within allowed iterations.")
  | otherwise               = newton f fP next tol (limit - 1)
  where
    next = guess - (f guess) / (fP guess)
\end{code}

%--------------------------------------------------------------------------------------------------
\section{M\"{u}ler's Method}
\paratag{SWRDD ZERO M\"{U}LER}
M\"{u}ler's method works by fitting a quadratic curve and using it to approximate the zero.  In order to make a quadratic curve, three points are needed.  M\"{u}ler's method will converge to a zero unless the three points all evaluate to the same value.  Since a quadratic curve is being approximated, complex zeros will also be found.
\subsection{Requirements}
\subsection{Function Definition}

\chapter{Main Program}
%--------------------------------------------------------------------------------------------------
\section{Program Headers}
\paratag{SWDDD MAIN HEADER}
The main part of the program is contained in the ``Main'' module.
\subsection{Requirements}
\begin{itemize}
\item The program \shalltag{10} be in a module named ``Main''.
\item The Main module \shalltag{20} import the Polynomial module.
\item The Main module \shalltag{30} import the Parse module.
\item The Main module \shalltag{40} import the Solver module.
\item Polynomials \shalltag{50} be stored in a list indexed by polynomial name.
\item The polynomial name \shalltag{60} be a String.
\item The polynomial \shalltag{70} be a list of double.
\end{itemize}

\subsection{Header Code}
\begin{code}
module Main where

import Char (isDigit, isSpace, isLower, toUpper)
import Text.Printf
import Polynomial
import Parse
import Solver
import Debug.Trace

type Name = String
type Poly = [Double]
type Entry = (String, [Double])
\end{code}

%--------------------------------------------------------------------------------------------------
\section{Parse Utilities}
\paratag{SWDDD MAIN UTIL}
\subsection{Requirements}
\begin{itemize}
\item The first word function \shalltag{10} return the first word of a line and the remainder of the line.
\item The uppercase function \shalltag{20} check if each character is lowercase, and if so, convert it to uppercase.
\item The find polynomial function \shalltag{30} search the polynomial library for a specified polynomial and return it.
\item If no the specified polynomial is not found, the find polynomial function \shalltag{40} return an empty list.
\end{itemize}

\subsection{Function Definition}
Function to convert a string to uppercase.
\begin{code}
upCase :: String -> String
upCase [] = []
upCase (x:xs)
  | isLower x = (toUpper x) : upCase xs
  | otherwise = x : upCase xs
\end{code}
Function to return the first word of a string, where words are separated by white space.
\begin{code}
firstWord :: String -> (String, String)
firstWord [] = ([], [])
firstWord (x:xs)
  | isSpace x = ([], xs)
  | otherwise = (x : cmd, rest)
    where
      (cmd, rest) = firstWord xs
\end{code}
Function to search the polynomial library for the name of a polynomial
\begin{code}
findPoly :: [(String, [Double])] -> String -> [Double]
findPoly [] _ = []

findPoly (x:xs) key
  | key == name = poly
  | otherwise   = findPoly xs key
  where
    (name, poly) = x
\end{code}

%--------------------------------------------------------------------------------------------------
\section{Command Dispatcher}
\paratag{SWDDD MAIN CMD}
\subsection{Requirements}
\begin{itemize}
\item The command dispatcher \shalltag{10} call the appropriate function for each command.
\item If an unrecognized command is given, an error message \shalltag{20} be produced.
\end{itemize}
\subsection{Function Definition}
\begin{code}
dispatch :: [Entry] -> String -> String -> ([Entry], String)
dispatch lib cmd rest
  | cmd == "ADD"       = binPoly lib rest polyAdd
  | cmd == "DEL"       = delPoly lib rest
  | cmd == "DERIVE"    = calcPoly lib rest polyDerive
  | cmd == "DIV"       = (lib, "Polynomial division not yet implemented.")
  | cmd == "EVAL"      = evaluate lib rest
  | cmd == "EXIT"      = (lib, "Exit command entered.")
  | cmd == "FACTOR"    = (lib, "Polynomial factoring not yet implemented.")
  | cmd == "HELP"      = (lib, help)
  | cmd == "INTEGRATE" = calcPoly lib rest polyIntegrate
  | cmd == "LIST"      = list lib
  | cmd == "MUL"       = binPoly lib rest polyMul
  | cmd == "NEW"       = newPoly lib rest
  | cmd == "QUIT"      = (lib, "Quit command entered.")
  | cmd == "ROOT"      = root lib rest
  | cmd == "TABLE"     = table lib rest
  | otherwise          = (lib, "<" ++ cmd ++ "> is not a valid command.")
\end{code}

%--------------------------------------------------------------------------------------------------
\section{Help Message}
\paratag{SWDDD MAIN HELP}
\subsection{Requirements}
\begin{itemize}
\item The help message \shalltag{10} contain a help message for the program.
\item The help message \shalltag{20} identify the valid commands, parameters, and function of the command.
\end{itemize}
\subsection{Function Definition}
\begin{code}
help :: String
help = "The following commands are defined.\n" ++
       "ADD P1 P2 P3\n" ++
       "   (add polynomials P1 and P2 together producing P3)\n" ++
       "DEL P1\n" ++
       "   (deletes polynomial named P1 from the saved polynomials)\n" ++
       "DERIVE P1 P2\n" ++
       "   (calculate the derivative of polynomial P1 and save as P2)\n" ++
       "DIV\n" ++
       "   (command not yet implemented)\n" ++
       "EVAL P1 VALUE\n" ++
       "   (calculate the value of polynomial P1 at VALUE)\n" ++
       "EXIT\n" ++
       "   (exit the program)\n" ++
       "FACTOR\n" ++
       "   (command not yet implemented)\n" ++
       "HELP\n" ++
       "   (produce this help message)\n" ++
       "INTEGRATE P1 P2\n" ++
       "   (calculate the integral of polynomial P1 and save as P2)\n" ++
       "LIST\n" ++
       "   (print a list of all saved polynomials)\n" ++
       "MUL P1 P2 P3\n" ++
       "   (multiply polynomials P1 and P2 together producing P3)\n" ++
       "NEW NAME POLY\n" ++
       "   (parse POLY as a polynomial and save it with the name NAME)\n" ++
       "QUIT\n" ++
       "   (quit the program)\n" ++
       "ROOT TYPE P1 OTHER\n" ++
       "   (find a root of P1 using method TYPE.  OTHER depends on type selected\n" ++
       "ROOT BISECT P1 LOWER UPPER\n" ++
       "   (use bisection method with LOWER and UPPER limits)\n" ++
       "ROOT NEWTON P1 GUESS TOL ITER\n" ++
       "   (use Newton's method with starting GUESS, tolerance TOL and iteration\n" ++
       "    limit ITER)\n" ++
       "TABLE P1 START STOP INCR\n" ++
       "   (print a table of values for P1 starting from START an continuing to\n" ++
       "    STOP with increment INCR)\n"
\end{code}
%--------------------------------------------------------------------------------------------------
\section{Command Loop}
\paratag{SWDDD MAIN LOOP}
\subsection{Requirements}
\begin{itemize}
\item The main loop \shalltag{10} print a prompt and request a line of input.
\item If the EXIT command is given, the main loop \shalltag{20} terminate.
\item All other commands \shalltag{30} be passed to the command dispatcher function
\end{itemize}
\subsection{Function Definition}
\begin{code}
mainLoop :: [Entry] -> IO ()

mainLoop lib = do
  putStr "> "
  line <- getLine
  let (cmd, rest) = firstWord (upCase line)
      (newLib, msg) = dispatch lib cmd rest
  putStrLn msg
  if cmd == "EXIT" || cmd == "QUIT"
  then return ()
  else mainLoop newLib
\end{code}

%--------------------------------------------------------------------------------------------------
\section{List Command}
\paratag{SWDDD MAIN LIST}
\subsection{Requirements}
\begin{itemize}
\item The list command \shalltag{10} list all the saved polynomials.
\item If no polynomials have been saved, an appropriate message \shalltag{20} be displayed.
\end{itemize}
\subsection{Function Definition}
Two functions are defined.  The list function checks if the library is null and prints an appropriate message.  If the library is not null, the temp list iterates through it and prints out the contents.
\begin{code}
list :: [Entry] -> ([Entry], String)
list [] = ([], "No polynomials have been saved.")
  
list lib = (lib, tempList lib)

tempList :: [Entry] -> String
tempList [] = ""
tempList (x:xs) = "<" ++ name ++ "> = <" ++ (polyShow poly) ++ ">\n" ++ tempList xs
  where (name, poly) = x
\end{code}

%--------------------------------------------------------------------------------------------------
\section{New Polynomial Command}
\paratag{SWDDD MAIN NEW}
\subsection{Requirements}
\begin{itemize}
\item The new command \shalltag{10} parse a polynomial and add it to the list.
\item Each polynomial added to the list \shalltag{20} have a unique name associated with it.
\item If the specified name is not unique, an error message \shalltag{30} be displayed and no new entry made.
\end{itemize}
\subsection{Function Definition}
The new polynomial function inserts a new polynomial into the library.  It defines three functions that are used locally.  Note that new entries are added to the beginning of the library list.
\begin{code}
newPoly :: [Entry] -> String -> ([Entry], String)
newPoly lib [] = (lib, "No polynomial or name specified")

newPoly lib rest = checkName lib name polyText
  where
    (name, polyText) = firstWord rest

    checkName :: [Entry] -> String -> String -> ([Entry], String)
    checkName lib name polyText
      | name == ""              = (lib, "No name specified for polynomial")
      | findPoly lib name /= [] = (lib, "Name <" ++ name ++ "> already exists")
      | otherwise               = checkPoly lib name polyText

    checkPoly ::  [Entry] -> String -> String -> ([Entry], String)
    checkPoly lib name polyText = insertPoly lib name polyList
      where
        polyList = parsePoly polyText

    insertPoly :: [Entry] -> String -> Poly -> ([Entry], String)
    insertPoly lib _ [] = (lib, "No valid polynomial specified.")
    insertPoly lib "" _ = (lib, "No polynomial name specified.")
    insertPoly lib name polyList = ((name, polyList) : lib,
      "Polynomial added to library.")
\end{code}

%--------------------------------------------------------------------------------------------------
\section{Delete Polynomial Command}
\paratag{SWDDD MAIN DEL}
\subsection{Requirements}
\begin{itemize}
\item The delete command \shalltag{10} remove the specified polynomial from the library.
\end{itemize}
\subsection{Function Definition}
\begin{code}
delPoly :: [Entry] -> String -> ([Entry], String)
delPoly lib [] = (lib, "No polynomial name specified.")

delPoly lib name
  | findPoly lib name == [] = (lib, "Polynomial <" ++ name ++ "> does not exists")
  | otherwise               = ((removePoly lib name), "Polynomial <" ++ name ++ "> deleted.")
  where
    removePoly ::  [Entry] -> String -> [Entry]
    removePoly [] name = []
    removePoly (l:ib) index
      | index == name = removePoly ib index
      | otherwise     = l : (removePoly ib index)
      where
        (name, _) = l
\end{code}


%--------------------------------------------------------------------------------------------------
\section{Binary Polynomial Commands}
\paratag{SWDDD MAIN BINARY}
\subsection{Requirements}
\begin{itemize}
\item The binary polynomial commands \shalltag{10} take two polynomials for input and produce a single polynomial output.
\item The input polynomials \shalltag{20} be specified by name in the polynomial library.
\item The output polynomial \shalltag{30} be specified as a name to be added to the polynomial library.
\end{itemize}
\subsection{Function Definition}
The type signature for the checkArg helper function is commented out because I couldn't find one that would work.
\begin{code}
binPoly :: [Entry] -> String -> ([Double] -> [Double] -> [Double]) -> ([Entry], String)
binPoly lib [] _ = (lib, "No polynomials specified.")

binPoly lib rest op = (newLib, msg)
  where
    (p1, rest1) = firstWord rest
    (p2, p3) = firstWord rest1
    (newLib, msg) = checkArg lib p1 p2 p3 op

--    checkArg :: [Entry] -> String -> String -> String -> ([Double] -> [Double] -> [Double]) -> ([Entry], String)
    checkArg lib "" _ _ _ = (lib, "No input polynomial specified.")
    checkArg lib _ "" _ _ = (lib, "No input polynomial specified.")
    checkArg lib _ _ "" _ = (lib, "No output polynomial specified.")
    checkArg lib p1 p2 p3 op
      | poly1 == [] = (lib, "Polynomial <" ++ p1 ++ "> not found.")
      | poly2 == [] = (lib, "Polynomial <" ++ p2 ++ "> not found.")
      | poly3 /= [] = (lib, "Polynomial <" ++ p3 ++ "> already exists.")
      | otherwise   = ((p3, (op poly1 poly2)) : lib, "New polynomial created.")
      where
        poly1 = findPoly lib p1
        poly2 = findPoly lib p2
        poly3 = findPoly lib p3
\end{code}

%--------------------------------------------------------------------------------------------------
\section{Calculus Polynomial Commands}
\paratag{SWDDD MAIN CALC}
\subsection{Requirements}
\begin{itemize}
\item The calculus polynomial commands \shalltag{10} take one polynomial for input and produce a single polynomial output.
\item The input polynomial \shalltag{20} be specified by name in the polynomial library.
\item The output polynomial \shalltag{30} be specified as a name to be added to the polynomial library.
\end{itemize}
\subsection{Function Definition}
The type signature for the checkArg helper function is commented out because I couldn't find one that would work.
\begin{code}
calcPoly :: [Entry] -> String -> ([Double] -> [Double]) -> ([Entry], String)
calcPoly lib [] _ = (lib, "No polynomials specified.")

calcPoly lib rest op = (newLib, msg)
  where
    (p1, p2) = firstWord rest
    (newLib, msg) = checkArg lib p1 p2 op

--    checkArg :: [Entry] -> String -> String -> ([Double] -> [Double] -> [Double]) -> ([Entry], String)
    checkArg lib "" _ _ = (lib, "No input polynomial specified.")
    checkArg lib _ "" _ = (lib, "No output polynomial specified.")
    checkArg lib p1 p2 op
      | poly1 == [] = (lib, "Polynomial <" ++ p1 ++ "> not found.")
      | poly2 /= [] = (lib, "Polynomial <" ++ p2 ++ "> already exists.")
      | otherwise   = ((p2, op poly1) : lib, "New polynomial created.")
      where
        poly1 = findPoly lib p1
        poly2 = findPoly lib p2
\end{code}

%--------------------------------------------------------------------------------------------------
\section{Evaluate Polynomial Command}
\paratag{SWDDD MAIN EVAL}
\subsection{Requirements}
\begin{itemize}
\item The evaluate polynomial command \shalltag{10} take one polynomial and a number for input and produce a single number output.
\item The input polynomial \shalltag{20} be specified by name in the polynomial library.
\item The specified polynomial \shalltag{30} be evaluated at the specified value.
\end{itemize}
\subsection{Function Definition}
\begin{code}
evaluate :: [Entry] -> String -> ([Entry], String)
evaluate lib [] = (lib, "No polynomials specified.")

evaluate lib rest = (newLib, msg)
  where
    (p1, value) = firstWord rest
    (newLib, msg) = checkArg lib p1 ((read value) :: Double)

    checkArg :: [Entry] -> String -> Double -> ([Entry], String)
    checkArg lib "" _ = (lib, "No input polynomial specified.")
    checkArg lib p1 value
      | poly1 == [] = (lib, "Polynomial <" ++ p1 ++ "> not found.")
      | otherwise   = (lib, msg)
      where
        poly1 = findPoly lib p1
        msg = "Value is " ++ (show (polyEval poly1 value))
\end{code}

%--------------------------------------------------------------------------------------------------
\section{Table Command}
\paratag{SWDDD MAIN TABLE}
\subsection{Requirements}
\begin{itemize}
\item The table commands \shalltag{10} take one polynomial, a starting value, an ending value, and an increment and produce a table.
\item The input polynomial \shalltag{20} be specified by name in the polynomial library.
\item The specified polynomial \shalltag{30} be evaluated at each value starting at the starting value, incrementing by the increment amount, until the ending value is reached.
\end{itemize}
\subsection{Function Definition}
The table function consists of a main function and three helper functions.
\begin{code}
table :: [Entry] -> String -> ([Entry], String)
table lib [] = (lib, "No polynomials specified.")

table lib rest = (newLib, msg)
  where
    (p1, r1) = firstWord rest
    (start, r2) = firstWord r1
    (end, incr) = firstWord r2
    (newLib, msg) = checkArg lib p1 ((read start) :: Double) ((read end) :: Double) ((read incr) :: Double)

    checkArg :: [Entry] -> String -> Double -> Double -> Double -> ([Entry], String)
    checkArg lib "" _ _ _ = (lib, "No input polynomial specified.")
    checkArg lib p1 start end incr
      | poly1 == [] = (lib, "Polynomial <" ++ p1 ++ "> not found.")
      | otherwise   = (lib, msg)
      where
        poly1 = findPoly lib p1
        msg = "  X  P(X)\n" ++ formatTable (tableHelp poly1 start end incr)

    tableHelp :: Poly -> Double -> Double -> Double -> [(Double, Double)]
    tableHelp [] _ _ _ = []
    tableHelp _ _ _ 0 = []
    tableHelp p1 start end incr
      | start > end = []
      | incr < 0    = []
      | otherwise   = (start, (polyEval p1 start)) : tableHelp p1 (start + incr) end incr

    formatTable :: [(Double, Double)] -> String
    formatTable [] = ""
    formatTable (x:xs) = (printf "%g  %g\n" y fy) ++ formatTable xs
      where
        (y, fy) = x
\end{code}

%--------------------------------------------------------------------------------------------------
\section{Root Command}
\paratag{SWDDD MAIN ROOTS}
This function uses the root finding functions from the Solver module to find zeros of a polynomial.  Note that currently, only real roots are supported.
\subsection{Requirements}
\begin{itemize}
\item The root finding function \shalltag{10} take a method, a polynomial, and one, or more initial guesses.
\item The bisection method \shalltag{20} take two initial guesses that must bracket the root.
\item The newton's method \shalltag{30} take one initial guess.
\end{itemize}

\subsection{Function Definition}
The root command determines which root finding algorithm has been specified and calls the appropriate algorithm.
\begin{code}
root :: [Entry] -> String -> ([Entry], String)
root lib [] = (lib, "No method or polynomial specified.")

root lib rest = (lib, msg)
  where
    (method, r1) = firstWord rest
    (p1, r2) = firstWord r1
    poly = findPoly lib p1
    msg = findMethod method poly r2
\end{code}
Determine which root finding method has been requested.
\begin{code}
    findMethod :: String -> Poly -> String -> String
    findMethod method poly rest
      | method == "BISECT" = findBisect poly rest
      | method == "NEWTON" = findNewton poly rest
      | poly == []         = "Polynomial not found in library."
      | otherwise          = "Method <" ++ method ++ "> is not a defined root finding method."
\end{code}
Wrapper for the bisection algorithm.  Parse out the upper and lower limits and call the bisection algorithm.
\begin{code}
    findBisect :: Poly -> String -> String
    findBisect poly rest
      | l == "" = "No lower limit specified."
      | u == "" = "No upper limit specified."
      | lower >= upper = "Lower limit must be less than upper limit."
      | otherwise = msg
      where
        (l, u) = firstWord rest
        lower = (read l) :: Double
        upper = (read u) :: Double
        f = polyEval poly
        (value, txt) = bisect f lower upper
        msg = "Zero value of " ++ (show value) ++ " found.\n" ++ txt
\end{code}
Wrapper for Newton's method algorithm.  Parse out the guess, tolerance, and iteration limit.  Determine derivative of the polynomial.  Call Newton's method.
\begin{code}
    findNewton :: Poly -> String -> String
    findNewton poly rest
      | g == "" = "No guess specified."
      | t == "" = "No tolerance specified."
      | l == "" = "No iteration limit specified."
      | otherwise = msg
      where
        (g, r) = firstWord rest
        (t, l) = firstWord r
        guess = (read g) :: Double
        tol = (read t) :: Double
        lim = (read l) :: Int
        derivative = polyDerive poly
        f = polyEval poly
        fP = polyEval derivative
        (value, txt) = newton f fP guess tol lim
        msg = "Zero value of " ++ (show value) ++ " found.\n" ++ txt
\end{code}

%--------------------------------------------------------------------------------------------------
\section{Main Function}
\paratag{SWDDD MAIN MAIN}
\subsection{Requirements}
\begin{itemize}
\item The main function \shalltag{10} print a header message when first entered.
\item The main function \shalltag{20} print a prompt for input.
\item The main function \shalltag{30} accept a line of input.
\item The line of input \shalltag{40} be converted to uppercase.
\item The main function \shalltag{50} pass the input line to the command parser for interpretation.
\item Polynomials \shalltag{60} be stored in an array that can be indexed by name.
\item The first word of the input line \shalltag{70} be interpreted as the command.
\item The ADD command \shalltag{80} add two specified polynomials and produce a third.
\item The DERIVE command \shalltag{90} calculate the derivative of the specified polynomial.
\item The DEL command \shalltag{100} delete a polynomial from the saved list.
\item The DIV command \shalltag{110} divide two specified polynomials and produce a third.
\item  The EVAL command \shalltag{120} compute the value of the specified polynomial at a specific value of X.
\item The EXIT command \shalltag{130} exit the program.
\item The FACTOR command \shalltag{140} produce the factors of a polynomial.
\item The HELP command \shalltag{150} produce a help message.
\item The INTEGRATE command \shalltag{160} produce the integral of the specified polynomial.
\item The LIST command \shalltag{170} produce a list of all the saved polynomials.
\item The MUL command \shalltag{180} multiply two specified polynomials and produce a third.
\item The NEW command \shalltag{190} create a new polynomial.
\item The ROOT command \shalltag{200} use various algorithms to find roots of the specified polynomial.
\item The TABLE command \shalltag{210} produce a table of values for the specified polynomial.
\item The QUIT command \shalltag{220} exit the program.
\end{itemize}

\subsection{Function Definition}
\begin{code}
main = do
  putStrLn "Simple symbolic math package."
  mainLoop [("P1", [-1,2,3]), ("P2", [3,2,1])]
\end{code}

%========================================================
%
%  Put various indices at the end of the document.
%
\addcontentsline{toc}{chapter}{Indices}
\printindex{tags}{List of Trace Tags}
\end{document}